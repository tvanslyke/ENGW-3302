\documentclass{article}

\usepackage{todonotes}
\usepackage{graphicx}
\usepackage{listings}
\usepackage{xcolor}
\usepackage{hyperref}
\hypersetup{
    colorlinks=true,
    linkcolor=blue,
    filecolor=magenta,      
    urlcolor=cyan,
}
\definecolor{lightlightgrey}{RGB}{240, 240, 240}
\definecolor{commentgrey}{RGB}{90, 90, 90}
\definecolor{typegreen}{RGB}{0, 90, 50}
\definecolor{keywordblue}{RGB}{50, 50, 200}
\definecolor{stringpurple}{RGB}{120, 50, 150}

\lstset{language=C++,
	basicstyle=\small\ttfamily,
	keywordstyle=\color{keywordblue},
	stringstyle=\color{stringpurple},
	commentstyle=\color{commentgrey},
	morecomment=[l][\color{magenta}]{\#},
	morekeywords={override, static_assert, nullptr, constexpr, auto, noexcept, decltype},
	morekeywords=[2]{string, tuple, vector, size_t, Animal, Cat, Dog, unique_ptr, shared_ptr, Derived, Base, ostream, ArrayList, Iterator, Integer, FILE, LinkedList},
	keywordstyle=[2]\color{typegreen},
	tabsize=4,
	frame=leftline, 
	framerule=4pt, 
	rulecolor=\color{lightlightgrey}, 
	framesep=0pt, 
	xleftmargin=5pt,
	backgroundcolor = \color{lightlightgrey},
}


\newcommand{\codefromfile}[1]{
	\lstinputlisting{#1}
}

\newcommand{\codeblock}[1]{
\begin{lstlisting}[frame=leftline, framerule=4pt, rulecolor=\color{gray}, framesep=15pt, xleftmargin=20pt]
#1
\end{lstlisting}
}

\newcommand{\uniqueptr}{\lstinline{std::unique\_ptr}}
\newcommand{\sharedptr}{\lstinline{std::shared\_ptr}}
\newcommand{\CppNew}{\lstinline{new}}
\newcommand{\CppDelete}{\lstinline{delete}}
\newcommand{\CppNewArray}{\lstinline{new[]}}
\newcommand{\CppDeleteArray}{\lstinline{delete[]}}
\newcommand{\NewAndDelete}{\CppNew{} and \CppDelete{}}

% font stuff
\usepackage{palatino}
\usepackage{helvet}
\usepackage[scaled]{beramono}
\usepackage[T1]{fontenc}

\newcommand{\placeholdertext}[1]{
	\noindent{\color{red}{#1}}
}

\title{An Open Letter to Educators Responsible for Teaching C++}
\author{An Enthusiastic C++ Programmer and Engineering Student}

\begin{document}
\raggedright
\maketitle

I'm primarily a hobbyist C++ programmer who likes to use bleeding-edge tools, who follows the the latest updates from the C++ standards committee, and who some might label as a ``language lawyer''.  At the same time, I'm currently studying electrical and computer engineering at Northeastern University in Boston.  In the just the past two years at NEU I've taken four courses that teach C++, all of which assume that students have little-to-no experience with the language.  Of these four courses, two suffer from the issues I discuss below, one teaches modern style, and the last is so extremely introductory that it is unfair to pass judgement on it.  In this letter, I will be simultaneously be drawing on my experiences as a student, active C++ community member, and occasional open-source C++ project contributor. 

I would also like to point out ahead of time that I have had only positive experiences with the teaching staff at Northeastern University.  In no way is this letter meant to imply that the quality of education at NEU is particularly lacking; this problem is \emph{not} exclusive to NEU and I'm most certainly not the first person to talk about this.  Two great talks that address problems in C++ education come from \href{https://www.youtube.com/watch?v=YnWhqhNdYyk}{Kate Gregory} and \href{https://www.youtube.com/watch?v=fX2W3nNjJIo}{Bjarne Stroustrup}.  Kate Gregory's talk is particularly effective and I would encourage not only educators but also professionals, hobbyists, and even students to watch it.

My primary concern that when C++ is improperly used as a means to teach other concepts it results in bad C++ programmers.  Some courses really ought to be using, for example, C as the teaching language because idiomatic C code more closely matches the content of the course.  \textbf{If you're going to teach a programming language as a side effect of teaching other concepts, choose one that fits the course content.  Doing otherwise risks teaching students bad practices.}  

C++ has found itself in the odd position of simultaneously being compatible with (mostly), more popular than (arguably), and newer than C.  Where C would have otherwise been chosen as a teaching language, C++ is instead.  Sometimes this is appropriate, sometimes not so much, but often the use of C++ looks like C with the additional use of the \lstinline{class}, \lstinline{new}, and \lstinline{virtual} keywords.  This style is commonly referred to as ``C with classes''; it is an outdated style superceded by not only the C++11 standard, but even C++98.  I primarilly take issue with courses that teach or promote this style of C++, particulary where its students have little prior experience with the language.

\section*{Modern C++ and Why You Should Teach it}
C++11 was a \emph{very} hefty update to a language that hadn't seen much love since 1998, the year the language was first standardized (C++03, was essentially just a mishmash of bug fixes).  The C++11 update introduced \emph{many} non-trivial changes that have drastically affected the way C++ is written today:  \lstinline{auto} type deduction, move semantics, lambdas, \lstinline{constexpr}, \lstinline{nullptr}, range-\lstinline{for}, \lstinline{enum class}, ``magic \lstinline{static}s'', uniform initialization syntax, perfect forwarding, \lstinline{using} type \& template aliases, \lstinline{decltype}, variadic templates, user-defined literals, \lstinline{static_assert}, \lstinline{noexcept}, \lstinline{[[attributes]]}, and explicitly \lstinline{default}ed/\lstinline{delete}d functions.  Additionally, library changes included a modern random number library, native concurrency support with threads \& asynchronous execution, \lstinline{std::unique_ptr} \& \lstinline{std::shared_ptr}, \lstinline{std::tuple} (made possible by variadic templates), hash tables \& sets, regular expressions, \lstinline{std::chrono}, and the \lstinline{type_traits} header.  C++11 was a huge game-changer for the language and has seen widespread adoption in all but the most firmly-entrenched legacy codebases.

Now I'm not saying that you should teach all or even most of the above features to new students; many of them are for fairly advanced use-cases.  Instead, I want to make it clear how much of a difference the new language standard has made.  Good C++11 code looks very different from good C++98/03 code.  Best practices in 2018 are not what they were a decade ago.  It's also important to note that the changes from C++11 (and C++14 \& C++17) are indicative of the direction that the C++ community is trying to move in\todo{fix this sentence}.  All of these changes happened because members of the C++ community proposed changes to the standard and a committee consisting largely of community members, academics, and representatives from interested firms (Google, Microsoft, Facebook, Bloomberg, \ldots) approved the changes.  The community wants C++ to be a safer, simpler language to program in.  Employers want their software engineers to be writing better, more maintainable code.  Teaching students bad, non-standard, or outdated practices does them a disservice, especially when they are left with the belief that the way they write C++ is perfectly acceptable and normal.  It's also unfair to the C++ community members who are working hard to make the language simpler and easier to use.

\section*{Teach \CppNew{} and \CppDelete{} Later}
% I actually only started picking up C++ around 2-3 years ago myself.  Like many others, my first experience with programming was an introductory course that was taught in Java and introduced OOP concepts.  This is a fairly common story, though Python has been gaining momentum as a teaching language too.  
\CppNew{} and \CppDelete{} are problematic components of the C++ core language.  They are the ``C++ way'' of allocating memory, supposedly making \lstinline{malloc()} and \lstinline{free()} obscelete.  In reality, \CppNew{} and \CppDelete{} are poor choices for anything other than single-object heap allocations where polymorphic behavior is desired:\todo{This flows horribly and is super hard to follow.  Fix it.}
\begin{lstlisting}
std::unique_ptr<Base> p_base{new Derived};
\end{lstlisting}
In C++14 we have \lstinline{std::make_unique()}, so this could even become:
\begin{lstlisting}
std::unique_ptr<Base> p_base{std::make_unique<Derived>()};
\end{lstlisting}

As a digression, it's worth pointing out that \NewAndDelete{} can be avoided entirely when teaching polymorphism\footnote{Kate Gregory advocates teaching polymorphism this way in her talk ``Stop Teaching C!''.}:

\begin{lstlisting}
// A classic OOP taxonomy example.
struct Animal {
	virtual ~Animal() = default;
	virtual std::string speak() const = 0;
};
struct Dog: public Animal {
	std::string speak() const override { return "Woof!"; }
};
struct Cat: public Animal {
	std::string speak() const override { return "Meow!"; }
};

// Polymorphic function.
void record_animal_speaking(const Animal& animal, std::ostream& os)
{ os << animal.speak(); }
int main() {
	Dog dog;
	Cat cat;
	// No new, delete, or even pointers!
	record_animal_speaking(dog, std::cout);
	std::cout << std::endl;
	record_animal_speaking(cat, std::cout);
	std::cout << std::endl;
}
\end{lstlisting}

When teaching students about \NewAndDelete{}, I would implore instructors to teach them alongside \lstinline{std::unique_ptr} and \lstinline{std::shared_ptr}.  This lesson should come \emph{after} having taught students about deterministic destruction (RAII), when they are already familiar with the concept of using scope and object lifetimes as a tool for managing resources.  When students are already familiar with this idea, \lstinline{std::unique_ptr} and \lstinline{std::shared_ptr} don't seem like ``magic''.  

I've encountered instructors who have referred to \lstinline{std::unique_ptr} as ``too advanced'' or ``too difficult'' to teach beginners.  I think that this is unfair.  When students are already familiar with RAII and scoped lifetimes (for e.g. closing files/connections/etc in destructors), the ``\NewAndDelete{} problem'' is easily recognized as being solvable in the exact same way\todo{this sentence is worded weirdly}.  This sets up the lesson nicely: introduce \NewAndDelete{}, stress that whenever an object created with \CppNew{} it must be \CppDelete{}'d before going out of scope (lest you introduce a memory leak), and then introduce \lstinline{std::unique_ptr} \& \lstinline{std::shared_ptr} as ready-made tools that handle the \CppDelete{}ing portion for them. 

Teaching \NewAndDelete{} this way will: 
\begin{itemize}
\item Give students a good sense of when and why they should use \NewAndDelete{}.
\item Reinforce the importance of and connection between scope, objective lifetimes, and resource management.
\item Teach students the modern and well-accepted way of managing the lifetime of heap-allocated objects. 
\item Avoid creating the false impression that heap allocation is needed to achieve polymorphic behavior.
\item Avoids hard-to-debug, memory-related bugs that can occur when students make mistakes while trying to do by-hand \CppNew{}s and \CppDelete{}s.
\end{itemize}

\subsection*{Smells Like Java}
There's a particular style of C++ programming I've seen that I like to call ``Java with pointers''\footnote{It's a play on the term ``C with classes''.  I'm hilarious.}.  ``Java with pointers'' is characterized by excessive use of the \lstinline{virtual} \& \lstinline{new} keywords, and reinventing the Java standard library with classes like \lstinline{ArrayList} that have methods like \lstinline{T ArrayList<T>::Get(int i);}.  In one course, we even implemented an interface called \lstinline{Iterator} that had \lstinline{virtual bool Iterator<T>::HasNext();} and \lstinline{virtual T Iterator<T>::Next();} methods.  That doesn't just smell like Java; that is exactly \texttt{java.util.Iterator}, modulo camel-case.

The thing is, C++ also has iterators.  In fact, iterators in C++ are a fundamental component of the STL.  However, they look \emph{nothing} like the \lstinline{Iterator} you would find in Java; Java's iterators more closely resemble the Iterator Pattern described in \textit{Design Patterns}.  C++'s iterators, on the other hand, are a set of \emph{concepts} that each model a subset of pointer-like behaviors.  They offer a much richer set of behaviors and are intended to allow for generic programming.  They are the basis of the STL, allowing C++ programmers to write generic code that works seamlessly with the rest of the standard library.

\begin{lstlisting}
// Erase all odd numbers in the container 'cont' using c++-style 
// iterators.
auto is_odd = [](auto number){
	return static_cast<bool>(number % 2);
};

// STL-style erase() member function for the container 
// takes iterators as arguments.
cont.erase(
	// pass iterators to the 'std::remove_if()' algorithm, 
	// which returns an iterator.
	remove_if(begin(cont), end(cont), is_odd), 
	end(cont)
);
\end{lstlisting}

\begin{lstlisting}
// Erase all odd numbers in the container 'cont' using Java-style
// iterators.  Here, of course, we have can't use standard library
// algorithms, since Java-style iterators don't satisfy the 
// iterator concept.
auto is_odd = [](auto number){
	return static_cast<bool>(number % 2);
};
// Heap allocation!  Gotta get that virtual dispatch!
std::unique_ptr<Iterator<int>> it(get_java_style_iterator(cont));
while(it->HasNext())
{
	if(is_odd(it->Next()))
		it->Remove();
}
\end{lstlisting}

The problem here is not that the ``Java-style'' code is less clear (it is less clear, but that's not the problem).  The problem is that iterators are a ``solved problem'', so to speak, in C++.  C++ programmers recognize and understand STL-style iterators. They communicate intent: ``I need to do some work that involves traversing this sequence of objects.''.  The ``Java-style'' iterator needlessly goes against the grain.  When a student goes to a job interview and is asked ``How would you use iterators to erase all odd numbers from a \lstinline{std::vector<int>}?'', you better bet your butt that the interviewer isn't looking for a Java-style solution.  If you're going to teach C++, then please don't fight the language; work \emph{with} it, not against it.

Now, let's talk about polymorphism and the \lstinline{virtual} keyword.  Dynamic dispatch is a useful tool and it's important that students learn how to use it.  It's even more important, in my opinion, that they learn \emph{when} and \emph{why} they should use it.  Sean Parent, a principle scientist at Adobe, gave a \textbf{fantastic} talk about this in 2017, entitled \href{https://www.youtube.com/watch?v=QGcVXgEVMJg}{Better Code: Runtime Polymorphism}\footnote{Sean has given quite a few talks about ``Better Code''.  I recommend them to anybody interested in becoming a better C++ programmer.}.  His main point is that inheritence is intrusive and leads to tight coupling.  Excessive use of inheritence and virtual functions makes for brittle code.  

That doesn't necessarilly mean that it's a Bad Thing; dynamic dispatch is a powerful tool!  However, unlike Java, C\#, and Python, where every member function is always ``virtual'', in C++ we have a choice.  Not every problem needs to be solved with an inheritence heirarchy.  There are many different ways to sort an array but that doesn't mean that we need add \lstinline{Sorter}, \lstinline{MergeSorter}, and \lstinline{QuickSorter} types to our program just so that we can dynamically change our sorting algorithm.  I posit that 99\% of the time, I have enough information at the call sight to choose my sorting algorithm appropriately for the data I have.  I can just call \lstinline{quick_sort()} directly if I know that the algorithm will likely suit the data; no dynamic dispatch and no ``\lstinline{Sorter}'' objects that just add noise to the code.  Sorting just isn't a very good use-case for polymorphism.

You know what is though?  I/O.  IOStreams, love em or hate em, are actually a pretty neat abstraction: 
\begin{lstlisting}
template <class Integer>
void write_bit_pattern(Integer number, std::ostream& output_stream)
{
	static_assert(std::is_integral_v<Integer>);
	constexpr auto bit_count = sizeof(Integer) * CHAR_BIT;
	std::bitset<bit_count> bit_pattern{number};
	output_stream << bit_pattern; 
}
\end{lstlisting}

That function could print the bit sequence to \lstinline{stdout} with \lstinline{std::cout}, or to a string with \lstinline{std::ostringstream}, or to a file with \lstinline{std::ofstream}.  Streams are a good candidate for polymorphism because we usually don't care about the actual target of an input/output operation at the call site.  It's a good abstraction and that's important.  When teaching students about polymorphism, try to find a good motivation for it!  Solving every problem with an inheritence heirarchy and dynamic dispatch just isn't good engineering.  The ``why'' is just as important as the ``how'', if not more so.

\section*{Maybe You Should be Teaching C}
C++ has a reputation of being a ``low-level'' language and is frequently used as a teaching langauge in courses that explore subjects such as memory management, low-level data structures, embedded design, and operating systems.  While I personally believe that C++ is \emph{usually} a better choice than C for solving most problems, using C++ to teach these concepts frequently leads to adoption of the infamous ``C with classes'' style of C++ programming.  It's important to remember that C and C++ are separate programming languages\footnote{Furthermore, there is no programming language called ``C/C++''.  I would implore any readers who use this term to erase it from their vocabulary immediately.} that encourage different style of programming and are suited to solving different sets of problems.

If your course's learning goals unavoidably mandate that you eschew modern C++ style (e.g. manual memory management), then C might be a better fit.  That is, you should prefer making good C programmers to making subpar C++ programmers.  Understanding how to write good C is extremely valuable in its own right.  For example, learning C has probably done more for my understanding of the importance and motivation behind object-oriented programming than learning C++ or Java has.  That's because in C, you have to follow conventions to enforce OOP.  Consider this collection of functions from the C standard library:
\begin{lstlisting}
FILE* fopen(const char* filename, const char* mode);
int fclose(FILE* stream);
int fputs(const char* str, FILE* stream);
char* fgets(char* str, int count, FILE* stream);
\end{lstlisting}

To the untrained eye, this is just a collection of C functions related to file IO.  What a shining example of messy, non-object oriented code!  Except, what if\ldots
\begin{lstlisting}
// constructor 
FILE* fopen(const char* filename, const char* mode);
// destructor
int fclose(FILE* this);
// member functions
int fputs(const char* str, FILE* this);
char* fgets(char* str, int count, FILE* this);
\end{lstlisting}
Hmmm\ldots
\begin{lstlisting}
class FILE {
	FILE(const char* filename, const char* mode); // fopen()
	~FILE(); // fclose()
	int puts(const char* str); // fputs()
	char* gets(char* str, int count); // fgets()
};
\end{lstlisting}

What an insight!  The \lstinline{<stdio.h>} header in C is actually a half-decent, if not occasionally inconsistent, example of object-oriented design!  In fact, that's how a lot of C code is written.  The CPython C API is an \emph{excellent} example of this, as are several components of the Linux kernel.  Now maybe I'm old fashioned, but I think this is worth learning.  

At NEU, I took a course where we had an assignment to design a \lstinline{LinkedList} class just like this.  There's a \lstinline{struct LinkedList} and a collection of functions that look like \lstinline{int LinkedList_MethodName(struct LinkedList* self, ...)}.  However, the course was taught in C++.  The very next assignment had us redesign the class using member functions, \lstinline{private} member variables, and all the other built-in OO stuff that you'd learn in any run-of-the-mill C++/C\#/Java course.  The idea behind designing the assignments this way is to show how to use C++'s OO features and how they make life easier with respect to doing it manually (i.e. ``the C way'').

I think it's a valuable lesson, but teaching it so directly is misguided.  The course was actually an introduction to embedded design.  Students learn how to use a Unix system, GCC, make, and C++\footnote{That's only half of the course, the other half focuses on computer architecture, boolean logic, etc.}.  The C++ component of the course really just boiled down to ``C with classes'' plus templates and \lstinline{virtual} member functions.  No STL (though there was plenty of the C standard library), no exceptions, no references (plenty of pointers though!).

The course missed out on an opportunity to make good C programmers who understand how to design an object-oriented API and effectively manage resources.  Instead the course taught students how \emph{not} to design an object-oriented API in C++ (i.e. ``the C way''), and a way that nobody really writes C++ anymore (``the C++98/pre C++98/C with classes way'').  In cases like this, a decision really has to be made about what language suits the course because, if you just use C++ because ``it's the newer one'' or ``it's more popular'' then you'll likely end up with students who have a warped unserstanding of what C++ programming is all about.

\section*{Designing a Course With C++}
Programming languages are tools.  Some courses are dedicated to teaching students how to use the tools themselves. For example, I took a dedicate \emph{Programming in C++} course at NEU.  It was awesome, we learned about \lstinline{auto} type deduction, lambdas, \lstinline{constexpr}, \lstinline{nullptr}, range-\lstinline{for}, uniform initialization syntax, \lstinline{using} type \& template aliases, \lstinline{decltype}, templates, \lstinline{static_assert}, \lstinline{noexcept}, and explicitly \lstinline{default}ed/\lstinline{delete}d functions.  Additionally, we learned how to use the STL to generate random numbers, write concurrent algorithms with \lstinline{std::thread} \& \lstinline{std::async}, manage object lifetimes with \lstinline{std::unique_ptr} \& \lstinline{std::shared_ptr} (and when you should choose each), \lstinline{std::tuple}, \lstinline{std::chrono}, and of course the STL containers and algorithms that have been a staple of C++ programming since 1998.  Of course other courses only use the tool to teach concepts like algorithm + data structure design, embedded design, or object oriented programming.  Such courses certainly shouldn't be expected to cover all of the same content that a dedicated C++ course would.  However, they \emph{absolutely should} be expected to not mislead students.

We should be careful when teaching C++.  There's no reason for students to come out of your course thinking ``C++ is hard because <manual memory management/you have to write everything from scratch/it's hard to debug segfaults>''.  C++ is all about \emph{not} manually managing resource, leveraging powerful and well-written libraries (\emph{particularly} the STL), and \textbf{preventing run-time bugs with static, compile-time facilities}.  Not teaching modern tools to students makes them worse C++ programmers, and frequently leads to them disliking the language.  

\subsection*{Find Motivations for Tools and Features}
I've addressed this previously, but I think it's particularly important.  Don't just teach the mechanics of a feature or library component, find a problem that it addresses and then show how it can be used to solve that problem.  For example, the 



\end{document}











