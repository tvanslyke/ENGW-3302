\documentclass{article}

\usepackage{todonotes}
\usepackage{graphicx}
\usepackage{listings}
\usepackage{xcolor}
\usepackage{hyperref}
\hypersetup{
    colorlinks=true,
    linkcolor=blue,
    filecolor=magenta,      
    urlcolor=cyan,
}
\definecolor{lightlightgrey}{RGB}{240, 240, 240}
\definecolor{commentgrey}{RGB}{90, 90, 90}
\definecolor{typegreen}{RGB}{0, 90, 50}
\definecolor{keywordblue}{RGB}{50, 50, 200}
\definecolor{stringpurple}{RGB}{120, 50, 150}

\lstset{language=C++,
	basicstyle=\small\ttfamily,
	keywordstyle=\color{keywordblue},
	stringstyle=\color{stringpurple},
	commentstyle=\color{commentgrey},
	morecomment=[l][\color{magenta}]{\#},
	morekeywords={override, static_assert, nullptr, constexpr, auto, noexcept, decltype},
	morekeywords=[2]{string, tuple, vector, size_t, Animal, Cat, Dog, unique_ptr, shared_ptr, Derived, Base, ostream, ArrayList, Iterator},
	keywordstyle=[2]\color{typegreen},
	tabsize=4,
	frame=leftline, 
	framerule=4pt, 
	rulecolor=\color{lightlightgrey}, 
	framesep=0pt, 
	xleftmargin=5pt,
	backgroundcolor = \color{lightlightgrey},
}


\newcommand{\codefromfile}[1]{
	\lstinputlisting{#1}
}

\newcommand{\codeblock}[1]{
\begin{lstlisting}[frame=leftline, framerule=4pt, rulecolor=\color{gray}, framesep=15pt, xleftmargin=20pt]
#1
\end{lstlisting}
}

\newcommand{\uniqueptr}{\lstinline{std::unique\_ptr}}
\newcommand{\sharedptr}{\lstinline{std::shared\_ptr}}
\newcommand{\CppNew}{\lstinline{new}}
\newcommand{\CppDelete}{\lstinline{delete}}
\newcommand{\CppNewArray}{\lstinline{new[]}}
\newcommand{\CppDeleteArray}{\lstinline{delete[]}}
\newcommand{\NewAndDelete}{\CppNew{} and \CppDelete{}}

% font stuff
\usepackage{palatino}
\usepackage{helvet}
\usepackage[scaled]{beramono}
\usepackage[T1]{fontenc}

\newcommand{\placeholdertext}[1]{
	\noindent{\color{red}{#1}}
}

\title{An Open Letter to Educators Responsible for Teaching C++}
\author{John Doe}

\begin{document}
\raggedright
\maketitle

\section*{The Problem}
Many university-level, introductory courses in C++ teach an outdated style that is unacceptable in 2018.  C++11 was a big game changer for the language that introduced \emph{\href{https://en.wikipedia.org/wiki/C\%2B\%2B11}{many}} new features that have very drastically changed the way we write C++ today.  In spite of this, many courses that teach C++, particularly courses where C++ is used as a vessel for teaching other programming concepts (e.g. algrothims and data structures), teach it as though we're still living in the previous millenium (C++98).

\section*{About Me and Other Introductory Stuff}
I'm primarily a hobbyist C++ programmer who likes to use bleeding-edge tools, who follows the the latest updates from the C++ standards committee, and who some might label as a ``language lawyer''.  At the same time, I'm currently studying electrical and computer engineering at Northeastern University in Boston.  In the just the past two years at NEU I've taken four courses that teach C++, all of which assume that students have little-to-no experience with the language.  Of these four courses, two suffer from teaching C++98isms, one teaches modern style, and the last is so extremely introductory that it is unfair to pass judgement on it.  In this letter, I will be simultaneously be drawing on my experiences as a student, active C++ community member, and occasional open-source C++ project contributor. 

I would also like to point out ahead of time that I have had only positive experiences with the teaching staff at Northeastern University.  In no way is this letter meant to imply that the quality of education at NEU is particularly lacking; this problem is \emph{not} exclusive to NEU and I'm most certainly not the first person to talk about this.  Two great talks that address problems in C++ education come from \href{https://www.youtube.com/watch?v=YnWhqhNdYyk}{Kate Gregory} and \href{https://www.youtube.com/watch?v=fX2W3nNjJIo}{Bjarne Stroustrup}.  \todo{Clean this sentence up.}I would strongly encourage any educators who are reading this to watch them, especially if you find arguments coming from a student to be unconvincing.


\todo{This needs reorganization and refinement.} The main point here is that when C++ is improperly used as a means to teach other concepts it results in bad C++ programmers.  Some courses really ought to be using, for example, C as the teaching language because idiomatic C code more closely matches what the content of the course addresses.  \textbf{If you're going to teach a programming language as a side effect of teaching other concepts, choose one that fits the course content.  Doing otherwise risks teaching students bad practices.}

\section*{Modern C++ and Why You Should Teach it}
C++11 was a \emph{very} hefty update to a language that hadn't seen much love since 1998, the year the language was first standardized (C++03, was essentially just a mishmash of bug fixes).  The C++11 update introduced \emph{many} non-trivial changes that have drastically affected the way C++ is written today:  \lstinline{auto} type deduction, move semantics, lambdas, \lstinline{constexpr}, \lstinline{nullptr}, range-\lstinline{for}, \lstinline{enum class}, ``magic \lstinline{static}s'', uniform initialization syntax, perfect forwarding, \lstinline{using} type \& template aliases, \lstinline{decltype}, variadic templates, user-defined literals, \lstinline{static_assert}, \lstinline{noexcept}, \lstinline{[[attributes]]}, and explicitly \lstinline{default}ed/\lstinline{delete}d functions.  Additionally, library changes included a modern random number library, native concurrency support with threads \& asynchronous execution, \lstinline{std::unique_ptr} \& \lstinline{std::shared_ptr}, \lstinline{std::tuple} (made possible by variadic templates), hash tables \& sets, regular expressions, \lstinline{std::chrono}, and the \lstinline{type_traits} header.  C++11 was a huge game-changer for the language and has seen widespread adoption in all but the most firmly-entrenched legacy codebases.

Now I'm not saying that you should teach all or even most of the above features to new students; many of them are for fairly advanced use-cases.  Instead, I want to make it clear how much of a difference the new language standard has made.  Good C++11 code looks very different from good C++98/03 code.  Best practices in 2018 are not what they were a decade ago.  It's also important to note that the changes from C++11 (and C++14 \& C++17) are indicative of the direction that the C++ community is trying to move in\todo{fix this sentence}.  All of these changes happened because members of the C++ community proposed changes to the standard and a committee consisting largely of community members, academics, and representatives from interested firms (Google, Microsoft, Facebook, Bloomberg, \ldots) approved the changes.  The community wants C++ to be a safer, simpler language to program in.  Employers want their software engineers to be writing better, more maintainable code.  Teaching students bad, non-standard, or outdated practices does them a disservice, especially when they are left with the belief that the way they write C++ is perfectly acceptable and normal.  It's also unfair to the C++ community members who are working hard to make the language simpler and easier to use.

\subsection*{Teach \CppNew{} and \CppDelete{} Later}
% I actually only started picking up C++ around 2-3 years ago myself.  Like many others, my first experience with programming was an introductory course that was taught in Java and introduced OOP concepts.  This is a fairly common story, though Python has been gaining momentum as a teaching language too.  
\CppNew{} and \CppDelete{} are problematic components of the C++ core language.  They are the ``C++ way'' of allocating memory, supposedly making \lstinline{malloc()} and \lstinline{free()} obscelete.  In reality, \CppNew{} and \CppDelete{} are poor choices for anything other than single-object heap allocations where polymorphic behavior is desired:
\begin{lstlisting}
std::unique_ptr<Base> p_base{new Derived};
\end{lstlisting}
In C++14 we have \lstinline{std::make_unique()}, so this could even become:
\begin{lstlisting}
std::unique_ptr<Base> p_base{std::make_unique<Derived>()};
\end{lstlisting}

As a digression, it's worth pointing out that \NewAndDelete{} can be avoided entirely when teaching polymorphism\footnote{Kate Gregory advocates teaching polymorphism this way in her talk ``Stop Teaching C!''.}:

\begin{figure}[h]
	\centering
	\codefromfile{code/polymorphism-no-pointers.cpp}
	% framesep=20pt, framexrightmargin=10pt
\end{figure}

When teaching students about \NewAndDelete{}, I would implore instructors to teach them alongside \lstinline{std::unique_ptr} and \lstinline{std::shared_ptr}.  This lesson should come \emph{after} having taught students about deterministic destruction (RAII), when they are already familiar with the concept of using scope and object lifetimes as a tool for managing resources.  When students are already familiar with this idea, \lstinline{std::unique_ptr} and \lstinline{std::shared_ptr} don't seem like ``magic''.  

I've encountered instructors who have referred to \lstinline{std::unique_ptr} as ``too advanced'' or ``too difficult'' to teach beginners.  I think that this is unfair.  When students are already familiar with RAII and scoped lifetimes (for e.g. closing files/connections/etc in destructors), the ``\NewAndDelete{} problem'' is easily recognized as being solvable in the exact same way\todo{this sentence is worded weirdly}.  This sets up the lesson nicely: introduce \NewAndDelete{}, stress that whenever an object created with \CppNew{} it must be \CppDelete{}'d before going out of scope (lest you introduce a memory leak), and then introduce \lstinline{std::unique_ptr} \& \lstinline{std::shared_ptr} as ready-made tools that handle the \CppDelete{}ing portion for them. 

Teaching \NewAndDelete{} this way will: 
\begin{itemize}
\item Give students a good sense of when and why they should use \NewAndDelete{}.
\item Reinforce the importance of and connection between scope, objective lifetimes, and resource management.
\item Teach students the modern and well-accepted way of managing the lifetime of heap-allocated objects. 
\item Avoid creating the false impression that heap allocation is needed to achieve polymorphic behavior.
\item Avoids hard-to-debug, memory-related bugs that can occur when students make mistakes while trying to do by-hand \CppNew{}s and \CppDelete{}s.
\end{itemize}

\subsection*{Teach the Standard Library Early}

This leads nicely into the topic of teaching the standard library or STL\footnote{Technically the STL $\neq$ the standard library but I, like many others, choose to ignore this distinction because nobody actually cares.}.  Typically (in the courses I have taken), \lstinline{std::string}s are introduced very early on.  This is good!  However, other core components of the STL are not introduced until towards the end of the course.

In my experience, neglecting the STL typically occurs because courses either 

For many students, \CppNewArray{} is the first way 


\subsection*{Smells Like Java}
There's a particular style of C++ programming I've seen that I like to call ``Java with pointers''\footnote{It's a play on the term ``C with classes''.  I'm hilarious.}.  ``Java with pointers'' is characterized by excessive use of the \lstinline{virtual} \& \lstinline{new} keywords, and reinventing the Java standard library with classes like \lstinline{ArrayList} that have methods like \lstinline{T ArrayList<T>::Get(int i);}.  In one course, we even implemented an interface called \lstinline{Iterator} with methods like \lstinline{bool Iterator<T>::HasNext();} and \lstinline{T Iterator<T>::Next();}.  That doesn't just smell like Java; that is exactly \texttt{java.util.Iterator}, modulo camel-case.

The thing is, C++ also has iterators.  In fact, iterators in C++ are a fundamental component of the STL.  However, they look \emph{nothing} like the \lstinline{Iterator} you would find in Java; Java's iterators more closely resemble the Iterator Pattern from \texttt{Design Patterns}.  C++'s iterators are a set of \emph{concepts} that each model a subset of pointer-like behaviors.  They offer a much richer set of behaviors and are intended to allow for generic programming.  They are the basis of the STL, allowing C++ programmers to write generic code that works seamlessly with the rest of the standard library. 

\begin{lstlisting}
// Erase all odd numbers in the container 'cont' using c++-style 
// iterators.
auto is_odd = [](auto number){
	return static_cast<bool>(number % 2);
};
cont.erase(
	remove_if(begin(cont), end(cont), is_odd),
	end(cont)
);
\end{lstlisting}

\begin{lstlisting}
// Erase all odd numbers in the container 'cont' using Java-style
// iterators.  Here, of course, we have can't use standard library
// algorithms, since Java-style iterators don't satisfy the 
// iterator concept.
auto is_odd = [](auto number){
	return static_cast<bool>(number % 2);
};
std::unique_ptr<Iterator<int>> it(get_java_style_iterator(cont));
while(it->HasNext())
{
	if(is_odd(it->Next()))
		it->Remove();
}
\end{lstlisting}

The problem here isn't that the ``Java-style'' code is less clear (it is less clear, but that's not the problem), it's that it 

\subsection{I Think Maybe You Should be Teaching C}
One of the courses that I took at NEU ha





\subsection*{Teach the Standard Library!}
\placeholdertext{Address the following:}
\begin{itemize}
	\item Vocabulary types like \texttt{std::vector}, \texttt{std::string}, \ldots
	\item In an ``Algorithms and Data Structures''-like course, \emph{please} follow the STL conventions.
	\begin{itemize}
		\item Leave virtual functions out of it!  This isn't Java, leave it for the OOP courses.
		\item It's okay to reinvent the wheel if you want to teach how wheels work; just make sure that students know how to use the wheels in the STL (vocabulary types are important!)
	\end{itemize}
	\item Address the ``the STL is too advanced for beginners'' misconception.
	\begin{itemize}
		\item \CppNew{}/\CppDelete{} are advanced tools (very easy to use them incorrectly).
		\item Code samples would help quite a bit here.
	\end{itemize}
\end{itemize}

\end{document}
