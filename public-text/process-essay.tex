\documentclass{article}

\usepackage{todonotes}
\usepackage{graphicx}
\usepackage{listings}
\usepackage{xcolor}
\usepackage{hyperref}
\hypersetup{
    colorlinks=true,
    linkcolor=blue,
    filecolor=magenta,      
    urlcolor=cyan,
}
\definecolor{lightlightgrey}{RGB}{240, 240, 240}
\definecolor{commentgrey}{RGB}{90, 90, 90}
\definecolor{typegreen}{RGB}{0, 90, 50}
\definecolor{keywordblue}{RGB}{50, 50, 200}
\definecolor{stringpurple}{RGB}{120, 50, 150}

\lstset{language=C++,
	basicstyle=\small\ttfamily,
	keywordstyle=\color{keywordblue},
	stringstyle=\color{stringpurple},
	commentstyle=\color{commentgrey},
	morecomment=[l][\color{magenta}]{\#},
	morekeywords={override, static_assert, nullptr, constexpr, auto, noexcept, decltype},
	morekeywords=[2]{string, tuple, vector, size_t, Animal, Cat, Dog, unique_ptr, shared_ptr, Derived, Base, ostream, ArrayList, Iterator, Integer, FILE, LinkedList},
	keywordstyle=[2]\color{typegreen},
	tabsize=4,
	frame=leftline, 
	framerule=4pt, 
	rulecolor=\color{lightlightgrey}, 
	framesep=0pt, 
	xleftmargin=5pt,
	backgroundcolor = \color{lightlightgrey},
}


\newcommand{\codefromfile}[1]{
	\lstinputlisting{#1}
}

\newcommand{\codeblock}[1]{
\begin{lstlisting}[frame=leftline, framerule=4pt, rulecolor=\color{gray}, framesep=15pt, xleftmargin=20pt]
#1
\end{lstlisting}
}

\newcommand{\uniqueptr}{\lstinline{std::unique\_ptr}}
\newcommand{\sharedptr}{\lstinline{std::shared\_ptr}}
\newcommand{\CppNew}{\lstinline{new}}
\newcommand{\CppDelete}{\lstinline{delete}}
\newcommand{\CppNewArray}{\lstinline{new[]}}
\newcommand{\CppDeleteArray}{\lstinline{delete[]}}
\newcommand{\NewAndDelete}{\CppNew{} and \CppDelete{}}

% font stuff
\usepackage{palatino}
\usepackage{helvet}
\usepackage[scaled]{beramono}
\usepackage[T1]{fontenc}

\newcommand{\placeholdertext}[1]{
	\noindent{\color{red}{#1}}
}

\title{Process Essay for Public Text}
\author{Timothy VanSlyke}

\begin{document}
\raggedright
\maketitle
This project was probably the most difficult for me to make work out of all of them.  I actually enjoy writing quite a bit, but I tend to struggle with asserting my opinion.  I'm the kind of person who almost always avoids speaking in absolutes (see what I did there?); if my certainty about something is even a smidge under 100\%, you won't catch me rounding up, so to speak.  This letter has a very strong opinion about a subject that affects a fairly diverse group of intellegent.  There's going to be \emph{some} nuance to my audience I'm going to be at least a little bit wrong about \emph{somebody}'s use-case for C++.




%%% stuff here?

This letter saw a much more gradual revision process than my previous writing.  Many of the issues I wanted to address only applied to some of my audience.  My advice about possibly teaching C instead of C++ just plain doesn't work for some courses.  A course that revolves around object-oriented programming (OOP) could absolutely be taught in C, but it probably shouldn't.  However, teaching C in a course about the fundamentals of embedded development is very much appropriate.  There's some overlap there too because an embedded development course could still strive to teach a thing or two about OOP.

% making sure my point is clear "it makes bad C++ programmers"

Another issue was dealing with the different levels of expertise within my target audience of professors and trainers.  Many instructors may have never truly used C++ outside of an academic context. It wouldn't be unheard of for a graduate student whose only experience with the language comes from previous coursework to be teaching a C++ course.  At the same time, there's no doubt that many of my audience members could be called ``C++ gurus''.  There is also variance in how passionate instructors are about the language.  I myself know the Java programming language fairly well and I could certainly teach the mechanics of the language to a newcomer.  However, I also take issue with some components the language's design, am not well-versed in the newest developments in the Java community, and I make no attempts to actively improve my knowledge of the language.  There's no doubt in my mind that members of my audience feel similarly about C++.

Since the driving force behind my arguments is hinged on the idea of making better C++ programmers, I just might not be able to reach some members of my audience.  It's unfortunate, but once I accepted that fact, I was able to more clearly structre my arguments.  I can't optimize my letter for everybody, so I chose to optimize for those that truly care about making better programmers.  For example, I posit that some courses are better accompanied by a language like C, rather than C++.  I make that recommendation in the interest of \emph{not} making bad C++ programmers.  If teaching the style of programming that I'm pushing in my letter directly interferes with a course's learning goals, then it's better that the course just choose a different language altogether.




\end{document}











